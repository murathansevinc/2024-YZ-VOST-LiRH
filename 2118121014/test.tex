\begin{enumerate}[label=\textbf{\arabic*.}]
  \item \textbf{Veri Toplama ve İşleme Bileşeni:}
    \begin{itemize}
        \item Sensörler aracılığıyla pistin gerçek zamanlı haritasını oluşturacak verileri toplama.
        \item Kullanılacak kamera veya diğer algılayıcılarla çevrenin görüntülerini toplama.
        \item Toplanan verilerin ön işlemesini yapma: gürültüyü azaltma, boyutu düşürme, filtreleme vb.
    \end{itemize}
  \item \textbf{Yapay Zeka ve Öğrenme Bileşeni:}
    \begin{itemize}
        \item Veri odaklı algoritma seçimi: Reinforcement Learning (teşvik öğrenimi) algoritması kullanılacak.
        \item Modelin eğitilmesi: Toplanan veriler üzerinde modelin eğitilmesi, pisti tamamlama görevini öğrenmesi.
        \item Modelin güncellenmesi: Sürekli olarak yeni verilerle modelin güncellenmesi ve iyileştirilmesi.
    \end{itemize}
  \item \textbf{Karar ve Yönlendirme Bileşeni:}
    \begin{itemize}
        \item Model tarafından verilen kararların alınması ve işlenmesi.
        \item Arabanın mevcut konumu ve pist haritası dikkate alınarak bir sonraki adımın belirlenmesi.
        \item Arabanın hareketini yönlendirme: Hız, yönlendirme açısı vb. ayarlama.
    \end{itemize}
  \item \textbf{Simülasyon ve Test Bileşeni:}
    \begin{itemize}
        \item Geliştirilen sistemin simülasyon ortamında test edilmesi.
        \item Farklı senaryolarda ve çevresel koşullarda sistemin performansının değerlendirilmesi.
        \item Modelin güvenilirliğinin ve dayanıklılığının test edilmesi.
    \end{itemize}
  \item \textbf{Görselleştirme ve Raporlama Bileşeni:}
    \begin{itemize}
        \item Sistemin çalışmasıyla ilgili verilerin görselleştirilmesi.
        \item Performans metriklerinin ve sonuçların raporlanması.
        \item Karar vericilere veya paydaşlara sunumlar için raporlar hazırlama.
    \end{itemize}
\end{enumerate}