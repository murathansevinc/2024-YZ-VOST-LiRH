\documentclass[12pt,a4paper]{article}
\usepackage[T1]{fontenc}
\usepackage{natbib}
\usepackage{graphicx}
\usepackage{pdflscape}
\date{}
\title{GAN ile Beyin Tümörü Görüntüsü Oluşturma}
\author{Emre Akkaya}

\begin{document}
	\begin{figure}[t!]
		\centering
		\includegraphics{ksbu.jpg}
	\end{figure}
	\maketitle
	\maketitle
	\begin{center}
	\section*{Giriş}
	\end{center}


Bu projede, beyin tümörlerinin erken teşhisinde önemli bir rol oynayan bir yapay zeka modeli geliştirilmesi amaçlanmaktadır.Projeyi seçme amacım beyin tümörü olan MR görüntülerinden faydalanarak oluşabilecek MR görüntülerini bulmak. Beyin tümörleri, sağlık endüstrisinde ciddi bir sağlık sorunu oluşturmakta ve erken teşhis, tedavi sürecindeki başarıyı büyük ölçüde etkileyebilmektedir. Bu proje kapsamında, çeşitli görüntüleme teknikleri kullanılarak toplanan beyin MR görüntüleri üzerinde bir derin öğrenme modeli oluşturulacak ve eğitilecektir. Model, beyin tümörlerini algılamak ve sınıflandırmak için optimize edilecektir. Eğitim süreci tamamlandıktan sonra, modelin performansı dikkatle değerlendirilecek ve gerektiğinde iyileştirmeler yapılacaktır. Bu proje, sağlık endüstrisindeki tıbbi görüntüleme uygulamalarına değerli bir katkı sağlayarak, beyin tümörlerinin erken teşhisini desteklemeyi amaçlamaktadır.



	
	

	\begin{enumerate}
		\clearpage
		\item		
		
		
		\begin{center}
			\section*{Literatür Araştırması}			
		\end{center}		
		
	Beyin tümörü, beyindeki anormal hücre kitlelerinin iyi veya kötü huylu olarak oluşması ve büyümesidir. Çalışma kapsamında beyin tümörü manyetik rezonans görüntüleri üzerinde evrişimsel sinir ağları temelli derin öğrenme modelleri kullanılarak tümör sınıflarının tespit edilmesi amaçlanmıştır. Beyin tümörlerinin sınıflandırılması için derin öğrenme modellerinden AlexNet, VGG ve MobileNet kullanılmıştır. Kaggle platformu üzerinden açık kaynaklı olarak paylaşılan bir normal ve üç anormal olmak üzere dört sınıflı yapıya sahip olan bir beyin tümörleri veri seti kullanılmıştır. Anormal sınıflar, glioma, meningioma ve pituitary'dir. Veri seti üzerinde sınıflandırma öncesinde ön işlem ile veri artırma adımlarında; kontrast sınırlı uyarlanabilir histogram eşitleme, dikey ve yatay çevirme işlemleri uygulanmıştır. Bu işlemin ardından derin öğrenme modellerinin veri setine bağımlılığını analiz edebilmek, azaltmak ve tümör sınıflarının tespit edilebilmesi için veri seti farklı farklı eğitim, doğrulama ve test yüzdelerinde kullanılarak sınıflandırmalar gerçekleştirilmiştir.\cite{article_1214984} 
	
		
	
		
		
		
		
		
		
		
		
	Rakamlar Ne Diyor?
	Hindistan'da her yıl 40.000 - 50.000 hastaya beyin tümörü tanısı konuluyor. Bunların yüzde 20'si çocuk
	Ülkenin mevcut nüfus düzeyinde (1,417 milyar), bu yalnızca yüzde 0,0035'e Beyin Tümörü tanısı konulduğu anlamına geliyor!
	Tüm MR taramalarının %100 doğru sonuçlar ürettiğini varsayalım. Bu, her 10.000 MRI taraması için Beyin Tümörünü gösteren yalnızca 35 örnek aldığımız ve göstermeyen çok daha fazla örnek aldığımız anlamına gelir.
	Bu, Tıbbi verilere erişimdeki diğer sorunlarla birleştiğinde Sınıf Dengesizliği ve Önyargı gibi Makine Öğrenimi sorunlarına yol açacaktır.. \cite{article_1214984}
		
		
		
		
		
	
		\begin{landscape}
			\begin{figure}[htbp]
				\centering
				\includegraphics{adsız.png}
				\caption*{Gantt Şeması} % Gantt şeması başlığı
				\label{fig:gantt}
			\end{figure}
		\end{landscape}
		
		
			
				
	\end{enumerate}
	\begin{center}
	
		\section*{Methodlar}
		\begin{enumerate}
			\item  Üretken Modelleme
			Üretken modeller veya derin üretken modeller, örnekten temel veri dağılımını öğrenen bir derin öğrenme modelleri sınıfıdır. Bu modeller, verileri temel özelliklerine indirgemek veya yeni ve çeşitli özelliklere sahip yeni veri örnekleri oluşturmak için kullanılabilir.
			\item Üretken Rekabet Ağları
			Üretken çekişmeli ağlar, verilerin istatistiksel dağılımından veri örnekleri üreten örtülü olasılık modelleridir. Veri kümesi içindeki varyasyonları kopyalamak için kullanılırlar. İki ağın bir kombinasyonunu kullanırlar: üretici ve ayırıcı.
		
			\begin{enumerate}
				\section*{üretici ve tüketici Neler İçerir}
				\item Üretici, girdi olarak rastgele bir gürültüden oluşan bir vektör alır ve bu gürültüyü kullanarak gerçekçi veri üretir.
				\item  Tüketici, gerçek veri ve üretilen veri arasındaki farkı belirlemek için eğitilir.
				
				\begin{figure}[htbp]
					\centering
					\includegraphics[width=\linewidth]{uretici_tuketici.png}
					\caption{Üretici ve Tüketici GAN Modeli}
					\label{fig:uretici_tuketici}
				\end{figure}
				
							
				\begin{center}
					\section*{Veriler}
				\end{center}
				
				
				Bu çalışmada 512 adet beyin MR görüntüsü kullanılmıştır.Bu veriler tümörlü ve tümörsüz olarak 2 kısımdan oluşmuştur
				
				\begin{figure}[htbp]
					\centering
					\includegraphics[width=\linewidth]{Y3.jpg}
					\caption{Tümörlü Beyin}
					\label{fig:uretici_tuketici}
				\end{figure}
				
				
				
				
				\clearpage
				
				\begin{center}
					\section*{Beklenen Sonuçlar}
				\end{center}
			Projenin başarılı olması durumunda, beyin tümörlerinin erken teşhisinde önemli bir adım atılmış olacaktır. Derin öğrenme tabanlı proje sayesinde, kullanıcılar kendi beyin MR görüntülerini analiz ederek potansiyel tümörleri belirleyebilecek ve erken teşhis sürecinde önemli bir rol oynayabileceklerdir. Bu, tümörlerin erken evreler
			de tespit edilmesine ve tedaviye başlanmasına olanak tanıyarak hastaların sağlık durumunu iyileştirecek ve yaşam kalitesini artıracaktır. Ayrıca, proje kullanıcılarına tümörün türü, büyüklüğü ve olası riskleri hakkında bilgi sağlayarak, bilinçli tedavi kararları almalarına yardımcı olacaktır. Projenin başarısı, sağlık endüstrisindeki tıbbi teknolojilerin geliştirilmesi ve hastalara daha iyi hizmet sunulması açısından önemli bir örnek teşkil edecektir.
				
				
				
				
				
				
			\end{enumerate}
		\end{enumerate}
		
	
			
		
		
		
		
		
		
	\end{center}
	\bibliographystyle{plain}
	\bibliography{references}
	
	
\end{document}